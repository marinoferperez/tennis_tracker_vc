% CVPR 2022 Paper Template
% based on the CVPR template provided by Ming-Ming Cheng (https://github.com/MCG-NKU/CVPR_Template)
% modified and extended by Stefan Roth (stefan.roth@NOSPAMtu-darmstadt.de)

\documentclass[10pt,twocolumn,letterpaper]{article}

%%%%%%%%% PAPER TYPE  - PLEASE UPDATE FOR FINAL VERSION
%\usepackage[review]{cvpr}      % To produce the REVIEW version
\usepackage{cvpr}              % To produce the CAMERA-READY version
%\usepackage[pagenumbers]{cvpr} % To force page numbers, e.g. for an arXiv version

% Include other packages here, before hyperref.
\usepackage{graphicx}
\usepackage{amsmath}
\usepackage{amssymb}
\usepackage{booktabs}



\newcommand{\latex}{\LaTeX\xspace}
\newcommand{\tex}{\TeX\xspace}


% It is strongly recommended to use hyperref, especially for the review version.
% hyperref with option pagebackref eases the reviewers' job.
% Please disable hyperref *only* if you encounter grave issues, e.g. with the
% file validation for the camera-ready version.
%
% If you comment hyperref and then uncomment it, you should delete
% ReviewTempalte.aux before re-running LaTeX.
% (Or just hit 'q' on the first LaTeX run, let it finish, and you
%  should be clear).
\usepackage[pagebackref,breaklinks,colorlinks]{hyperref}


% Support for easy cross-referencing
\usepackage[capitalize]{cleveref}
\crefname{section}{Sec.}{Secs.}
\Crefname{section}{Section}{Sections}
\Crefname{table}{Table}{Tables}
\crefname{table}{Tab.}{Tabs.}
\setlength{\parindent}{1.5em}
\setlength{\parskip}{0.8em}

%%%%%%%%% PAPER ID  - PLEASE UPDATE
\def\cvprPaperID{*****} % *** Enter the CVPR Paper ID here
\def\confName{CVPR}
\def\confYear{2026}


\begin{document}

%%%%%%%%% TITLE - PLEASE UPDATE
\title{Sistema de Detección y Seguimiento Automático de Jugadores y Pelota en
Partidos de Tenis en Tiempo Real}

\author{
Marino Fernández Pérez\\
{\tt\small marinoferpe@correo.ugr.es}
\and
Francesc Oliver Catany\\
{\tt\small francescoliver@correo.ugr.es}
\and
Pau Bover Femenias\\
{\tt\small paubover@correo.ugr.es}
\and
Gabriele Ruggeri\\
{\tt\small gabricross37@correo.ugr.es}
\\[1em]
\textbf{Universidad de Granada (UGR)} - Visión por Computador
}


\maketitle

%%%%%%%%% Resumen 
\begin{abstract}
En este trabajo se presenta un sistema para el analisis automatico de un partido de tenis a partir de vídeo broadcast. El sistema es capaz de identificar la pista de juego, sus líneas y keypoints relevantes, así como detectar y seguir a los jugadores y la pelota a lo largo de la secuencia de vídeo. A partir de esta información, se estiman métricas cinemáticas como la velocidad y la distancia recorrida por los jugadores durante el partido.

La solución propuesta integra múltiples modelos de aprendizaje profundo junto con técnicas clásicas de visión por computador, organizados en un pipeline modular y extensible. El sistema ha sido evaluado en vídeos reales de diferentes competiciones y tipos de pista, mostrando un funcionamiento robusto en escenarios variados. Aunque el procesamiento se realiza actualmente en un régimen de casi tiempo real, se discute su viabilidad para aplicaciones en tiempo real mediante optimizaciones adicionales. Los resultados obtenidos demuestran el potencial del enfoque propuesto como herramienta de análisis automático en el ámbito del tenis profesional.
\end{abstract}

%%%%%%%%% BODY TEXT
\section{Introducción}
\label{sec:intro}

En el ámbito del tenis profesional, el análisis detallado del rendimiento tanto de jugadores propios como de rivales supone una tarea compleja y costosa para entrenadores y analistas. Gran parte de este análisis se realiza de forma manual o semiautomática, lo que dificulta la obtención de información objetiva y detallada a partir de grandes volúmenes de vídeo.

La resolución de este problema resulta relevante debido al creciente interés en el uso de herramientas basadas en inteligencia artificial para el análisis avanzado del rendimiento deportivo. La información recopilada por sistemas automáticos de detección y seguimiento permite desarrollar métricas complejas y estimaciones sobre el comportamiento de los jugadores durante el partido. Este tipo de análisis puede proporcionar una ventaja competitiva en el estudio de rivales, facilitando la identificación de patrones de juego y tendencias estratégicas. Además, este tipo de tecnología presenta un alto potencial de aplicación en retransmisiones televisivas, donde puede emplearse para ofrecer información visual y estadística de interés que enriquezca la experiencia de los espectadores.

El objetivo de este trabajo es desarrollar un sistema que permita detectar de manera automática información relevante a partir de partidos de tenis, de forma que cualquier usuario pueda utilizarlo para analizar vídeos y construir métricas y análisis avanzados basados en dicha información. 

\subsection{Motivación personal}

Elegimos este tema como proyecto porque nos pareció un campo especialmente amplio y estimulante, en el que era posible profundizar en muchos de los conceptos que más nos habían interesado a lo largo de la asignatura. El análisis de vídeo deportivo combina técnicas de aprendizaje profundo, procesamiento clásico de imágenes y geometría, lo que nos permitió aplicar de forma práctica conocimientos diversos que iremos detallando más adelante. Además, la complejidad real del problema y la necesidad de tomar decisiones de diseño fundamentadas lo convirtieron en un contexto idóneo para consolidar y ampliar lo aprendido durante el curso.


\section{Contexto}

El análisis automático de partidos de tenis a partir de vídeo plantea diversos retos desde el punto de vista de la visión por computador, como la detección y el seguimiento de los elementos dinámicos de la escena, así como la localización precisa de las estructuras geométricas estáticas de la pista. La correcta combinación de técnicas de aprendizaje profundo y métodos clásicos de procesamiento de imagen resulta clave para obtener resultados robustos y coherentes.

En los últimos años, los modelos de detección basados en deep learning han demostrado un gran rendimiento en tareas de localización de objetos en tiempo real. Modelos como YOLO permiten detectar de forma eficiente jugadores u otros elementos relevantes dentro de la escena, mientras que redes especializadas como TrackNet han sido diseñadas específicamente para el seguimiento de la pelota en secuencias de vídeo. Por su parte, arquitecturas ligeras como MobileNetV3-Small permiten realizar inferencia con un coste computacional reducido, lo que las hace especialmente adecuadas para sistemas en tiempo real o con recursos limitados, en este trabajo usaremos estos modelos para diferentes tareas.

Sin embargo, la detección de las líneas de la pista no se beneficia necesariamente de enfoques puramente basados en aprendizaje profundo, ya que se trata de estructuras geométricas bien definidas y con propiedades conocidas. En este contexto, los métodos clásicos de procesamiento de imagen siguen siendo una alternativa eficaz, con una buena refrencia base. La Transformada de Hough es una técnica ampliamente utilizada para la detección de líneas rectas. Sin una adecuada eliminación del fondo y del ruido, Hough tiende a generar detecciones espurias. Para mejorar la calidad geométrica de las líneas detectadas, tambien aplicamos algoritmos de afinado como el método de Zhang–Suen.

Finalmente, la relación entre la imagen capturada por la cámara y el plano real de la pista puede modelarse mediante técnicas de geometría proyectiva. La homografía permite establecer una correspondencia entre ambos planos, lo que posibilita la rectificación de la imagen, la superposición de modelos geométricos y la realización de mediciones espaciales. Para que esta transformación sea fiable, es imprescindible disponer de keypoints precisos y coherentes desde el punto de vista geométrico, lo que motiva la necesidad de refinar su posición a partir de las intersecciones reales de las líneas de la pista.

\section{Trabajos previos}

En los últimos años se han propuesto diversos enfoques para el análisis automático de partidos de tenis a partir de vídeo, abordando problemas como la detección de la pista, el seguimiento de la pelota y de los jugadores, así como la estimación geométrica de la escena. Estos trabajos combinan técnicas de aprendizaje profundo con métodos clásicos de visión por computador, sentando las bases sobre las que se apoya el presente proyecto.

Uno de los trabajos más relevantes es el presentado en Tennis Analytics: Tracking Ball and Players Using Deep Learning (2019), donde se propone un sistema completo para el análisis de partidos de tenis mediante redes neuronales profundas. En este trabajo se introduce el uso de TrackNet para el seguimiento de la pelota a lo largo del tiempo, demostrando que los modelos basados en secuencias temporales son especialmente adecuados para capturar trayectorias rápidas y de pequeño tamaño. Este enfoque pone de manifiesto la importancia de combinar información espacial y temporal para obtener resultados robustos en escenarios reales.

En el repositorio tennis\_analysis se presenta una implementación práctica de un sistema de análisis de tenis que integra detección de jugadores, seguimiento de la pelota y detección de líneas de la pista. Este trabajo destaca por su enfoque modular y experimental, mostrando cómo distintas técnicas pueden integrarse en un único pipeline funcional. En particular, se observa el uso de métodos clásicos para la detección de la pista combinados con modelos de aprendizaje profundo para los elementos dinámicos de la escena, lo que refuerza la idea de que un enfoque híbrido resulta especialmente efectivo.

Por otro lado, el proyecto TennisCourtDetector se centra específicamente en la detección de las líneas de la pista de tenis. Este trabajo explota las propiedades geométricas conocidas de la pista y emplea técnicas como la Transformada de Hough junto con reglas geométricas para filtrar y validar las líneas detectadas. Su principal aportación es demostrar que, con un preprocesado adecuado y un filtrado geométrico consistente, es posible obtener una detección fiable de la estructura de la pista sin recurrir exclusivamente a modelos de aprendizaje profundo.

Desde una perspectiva más orientada al aprendizaje automático, el artículo Improving Tennis Court Line Detection with Machine Learning explora el uso de modelos entrenados específicamente para segmentar las líneas de la pista. Este enfoque muestra una mayor robustez frente a variaciones de iluminación, sombras y oclusiones parciales, aunque introduce una mayor complejidad computacional y la necesidad de datos etiquetados. Este trabajo resulta especialmente relevante al evidenciar las ventajas y limitaciones de los métodos puramente basados en aprendizaje profundo frente a soluciones geométricas clásicas.

Finalmente, el trabajo Learning to Track: Online Multi-Object Tracking by Decision Making (2015) aporta conceptos fundamentales relacionados con el seguimiento de objetos en vídeo, estableciendo bases teóricas y prácticas que han influido en numerosos sistemas posteriores de tracking. Aunque no está centrado exclusivamente en el tenis, sus aportaciones son relevantes para entender los principios que subyacen al seguimiento robusto de objetos en escenas dinámicas.

En conjunto, estos trabajos previos han servido como referencia fundamental para el desarrollo de este proyecto. A partir de ellos, se ha optado por un enfoque híbrido que combina detección mediante aprendizaje profundo, procesamiento clásico de imágenes y geometría proyectiva. En lugar de adoptar una única solución existente, se han integrado y adaptado ideas de los distintos trabajos, experimentando con cada una de ellas para construir un sistema más robusto y ajustado a los objetivos planteados.\cite{niehorster2019tennis,abdullahtarek2022tennis, ml62021court,xiang2015learning,yastrebksv2021court}

\section{Methods}

Detailed description of the methods used and/or proposed, and clear justification of why these methods are used and not others.

\section{Experiments}

The data used, the experimental validation protocol, the metrics used, the experiments carried out, the results obtained, and their discussion are presented here.

\subsection{Dataset}

\section{Conclusions}

Section that presents, briefly and as a summary, the main conclusions of the work carried out. It also usually includes future possible works. That is, what are the most promising lines to continue with this work, as well as possible proposals for improvement. IMPORTANT: these are the scientific conclusions reached in the project; not your personal conclusions about the work you have done!



%%%%%%%%% REFERENCES
{\small
\bibliographystyle{ieee_fullname}
\bibliography{egbib}
}

\end{document}
