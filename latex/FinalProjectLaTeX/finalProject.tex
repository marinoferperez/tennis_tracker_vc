% CVPR 2022 Paper Template
% based on the CVPR template provided by Ming-Ming Cheng (https://github.com/MCG-NKU/CVPR_Template)
% modified and extended by Stefan Roth (stefan.roth@NOSPAMtu-darmstadt.de)

\documentclass[10pt,twocolumn,letterpaper]{article}

%%%%%%%%% PAPER TYPE  - PLEASE UPDATE FOR FINAL VERSION
%\usepackage[review]{cvpr}      % To produce the REVIEW version
\usepackage{cvpr}              % To produce the CAMERA-READY version
%\usepackage[pagenumbers]{cvpr} % To force page numbers, e.g. for an arXiv version

% Include other packages here, before hyperref.
\usepackage{graphicx}
\usepackage{amsmath}
\usepackage{amssymb}
\usepackage{booktabs}



\newcommand{\latex}{\LaTeX\xspace}
\newcommand{\tex}{\TeX\xspace}


% It is strongly recommended to use hyperref, especially for the review version.
% hyperref with option pagebackref eases the reviewers' job.
% Please disable hyperref *only* if you encounter grave issues, e.g. with the
% file validation for the camera-ready version.
%
% If you comment hyperref and then uncomment it, you should delete
% ReviewTempalte.aux before re-running LaTeX.
% (Or just hit 'q' on the first LaTeX run, let it finish, and you
%  should be clear).
\usepackage[pagebackref,breaklinks,colorlinks]{hyperref}


% Support for easy cross-referencing
\usepackage[capitalize]{cleveref}
\crefname{section}{Sec.}{Secs.}
\Crefname{section}{Section}{Sections}
\Crefname{table}{Table}{Tables}
\crefname{table}{Tab.}{Tabs.}
\setlength{\parindent}{1.5em}
\setlength{\parskip}{0.8em}

%%%%%%%%% PAPER ID  - PLEASE UPDATE
\def\cvprPaperID{*****} % *** Enter the CVPR Paper ID here
\def\confName{CVPR}
\def\confYear{2026}


\begin{document}

%%%%%%%%% TITLE - PLEASE UPDATE
\title{Sistema de Detección y Seguimiento Automático de Jugadores y Pelota en
Partidos de Tenis en Tiempo Real}

\author{
Marino Fernández Pérez\\
{\tt\small marinoferpe@correo.ugr.es}
\and
Francesc Oliver Catany\\
{\tt\small francescoliver@correo.ugr.es}
\and
Pau Bover Femenias\\
{\tt\small paubover@correo.ugr.es}
\and
Gabriele Ruggeri\\
{\tt\small gabricross37@correo.ugr.es}
\\[1em]
\textbf{Universidad de Granada (UGR)} - Visión por Computador
}


\maketitle

%%%%%%%%% Resumen 
\begin{abstract}
En este trabajo se presenta un sistema para el analisis automatico de un partido de tenis a partir de vídeo broadcast. El sistema es capaz de identificar la pista de juego, sus líneas y keypoints relevantes, así como detectar y seguir a los jugadores y la pelota a lo largo de la secuencia de vídeo. A partir de esta información, se estiman métricas cinemáticas como la velocidad y la distancia recorrida por los jugadores durante el partido.

La solución propuesta integra múltiples modelos de aprendizaje profundo junto con técnicas clásicas de visión por computador, organizados en un pipeline modular y extensible. El sistema ha sido evaluado en vídeos reales de diferentes competiciones y tipos de pista, mostrando un funcionamiento robusto en escenarios variados. Aunque el procesamiento se realiza actualmente en un régimen de casi tiempo real, se discute su viabilidad para aplicaciones en tiempo real mediante optimizaciones adicionales. Los resultados obtenidos demuestran el potencial del enfoque propuesto como herramienta de análisis automático en el ámbito del tenis profesional.
\end{abstract}

%%%%%%%%% BODY TEXT
\section{Introducción}
\label{sec:intro}

En el ámbito del tenis profesional, el análisis detallado del rendimiento tanto de jugadores propios como de rivales supone una tarea compleja y costosa para entrenadores y analistas. Gran parte de este análisis se realiza de forma manual o semiautomática, lo que dificulta la obtención de información objetiva y detallada a partir de grandes volúmenes de vídeo.

La resolución de este problema resulta relevante debido al creciente interés en el uso de herramientas basadas en inteligencia artificial para el análisis avanzado del rendimiento deportivo. La información recopilada por sistemas automáticos de detección y seguimiento permite desarrollar métricas complejas y estimaciones sobre el comportamiento de los jugadores durante el partido. Este tipo de análisis puede proporcionar una ventaja competitiva en el estudio de rivales, facilitando la identificación de patrones de juego y tendencias estratégicas. Además, este tipo de tecnología presenta un alto potencial de aplicación en retransmisiones televisivas, donde puede emplearse para ofrecer información visual y estadística de interés que enriquezca la experiencia de los espectadores.

El objetivo de este trabajo es desarrollar un sistema que permita detectar de manera automática información relevante a partir de partidos de tenis, de forma que cualquier usuario pueda utilizarlo para analizar vídeos y construir métricas y análisis avanzados basados en dicha información. 

\subsection{Motivación personal}

\section{Contexto}

This section presents the fundamental concepts necessary to understand the work.


\section{Related Works}

It presents what has been done in the field previously, and what the best methods are currently. It is very important, in general, not only in this section, to adequately document the relevant literature. To do this, you must use the $.bib$ file, in the way I show here: \cite{mesejo2016computer, lathuiliere2019comprehensive, vargas2023deep}

\section{Methods}

Detailed description of the methods used and/or proposed, and clear justification of why these methods are used and not others.

\section{Experiments}

The data used, the experimental validation protocol, the metrics used, the experiments carried out, the results obtained, and their discussion are presented here.

\subsection{Dataset}

\section{Conclusions}

Section that presents, briefly and as a summary, the main conclusions of the work carried out. It also usually includes future possible works. That is, what are the most promising lines to continue with this work, as well as possible proposals for improvement. IMPORTANT: these are the scientific conclusions reached in the project; not your personal conclusions about the work you have done!



%%%%%%%%% REFERENCES
{\small
\bibliographystyle{ieee_fullname}
\bibliography{egbib}
}

\end{document}